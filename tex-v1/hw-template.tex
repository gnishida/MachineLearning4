\documentclass[12pt,letterpaper]{article}

\usepackage{amsfonts}
\usepackage{graphics}
\usepackage{graphicx}
\usepackage{colortbl}
\usepackage{amsmath}

\usepackage{algorithm}
\usepackage[noend]{algpseudocode}
\usepackage{array}

\usepackage{epstopdf}

\setlength\intextsep{10pt}
\setlength\textfloatsep{10pt}

\newenvironment{proof}{\noindent{\bf Proof:}}{\qed\bigskip}

\newtheorem{theorem}{Theorem}
\newtheorem{corollary}{Corollary}
\newtheorem{lemma}{Lemma} 
\newtheorem{claim}{Claim}
\newtheorem{fact}{Fact}
\newtheorem{definition}{Definition}
\newtheorem{assumption}{Assumption}
\newtheorem{observation}{Observation}
\newtheorem{example}{Example}
\newcommand{\qed}{\rule{7pt}{7pt}}

\newcommand{\assignment}[4]{
\thispagestyle{plain} 
\newpage
\setcounter{page}{1}
\noindent
\begin{center}
\framebox{ \vbox{ \hbox to 6.28in
{\bf CS578/STAT590: Introduction Machine Learning \hfill #1}
\vspace{4mm}
\hbox to 6.28in
{\hspace{2.5in}\large\mbox{Problem Set #2}}
\vspace{4mm}
\hbox to 6.28in
{{\it Handed Out: #3 \hfill Due: #4}}
}}
\end{center}
}

\newcommand{\solution}[4]{
\thispagestyle{plain} 
\newpage
\setcounter{page}{1}
\noindent
\begin{center}
\framebox{ \vbox{ \hbox to 6.28in
{\bf CS578/STAT590: Introduction to Machine Learning \hfill #4}
\vspace{4mm}
\hbox to 6.28in
{\hspace{2.5in}\large\mbox{Problem Set #3}}
\vspace{4mm}
\hbox to 6.28in
{#1 \hfill {\it Handed In: #2}}
}}
\end{center}
\markright{#1}
}



\def\Comment#1{\textsf{\textsl{$\langle\!\langle$#1\/$\rangle\!\rangle$}}}



\oddsidemargin 0in
\evensidemargin 0in
\textwidth 6.5in
\topmargin -0.5in
\textheight 9.0in

\begin{document}

\solution{Gen Nishida}{\today}{4}{Fall 2014}
% Fill in the above, for example, as follows:
% \solution{John Smith}{\today}{1}{Fall 2014}

\pagestyle{myheadings}  % Leave this command alone

\section*{Questions}

\begin{enumerate}
\item {\bf Boosting}

\begin{enumerate}
\item[(1)] The weak hypothesis used at each round, and its error

{\bf 1st round} $h_1(x)={\rm sign} (x_1>6)$, $\epsilon_1=0.2$ ($\alpha_1=0.6931$)

\begin{table}[htb]
  \begin{center}
  \begin{tabular}{|c|c|c|c|c|} \hline
    index & $x_1$ & $x_2$ & $y$ & prediction \\ \hline
    1 & 1 & 10 & - & - \\ \hline
    2 & 4 & 4 & - & -\\ \hline
    3 & 8 & 7 & + & + \\ \hline
    4 & 5 & 6 & - & - \\ \hline
    5 & 3 & 16 & - & - \\ \hline
    6 & 7 & 7 & + & + \\ \hline
    7 & 10 & 14 & + & + \\ \hline
    8 & 4 & 2 & - & - \\ \hline
    9 & 4 & 10 & + & - \\ \hline
    10 & 8 & 8 & - & + \\ \hline
  \end{tabular}
  \caption{The prediction by $h_1$}
  \label{tab:1st_round_prediction}
  \end{center}
\end{table}

{\bf 2nd round} $h_2(x)=sign (x_2>9)$, $\epsilon_1=0.25$ ($\alpha_2=0.5493$)

\begin{table}[htb]
  \begin{center}
  \begin{tabular}{|c|c|c|c|c|} \hline
    index & $x_1$ & $x_2$ & $y$ & prediction \\ \hline
    1 & 1 & 10 & - & + \\ \hline
    2 & 4 & 4 & - & - \\ \hline
    3 & 8 & 7 & + & - \\ \hline
    4 & 5 & 6 & - & - \\ \hline
    5 & 3 & 16 & - & + \\ \hline
    6 & 7 & 7 & + & - \\ \hline
    7 & 10 & 14 & + & + \\ \hline
    8 & 4 & 2 & - & - \\ \hline
    9 & 4 & 10 & + & + \\ \hline
    10 & 8 & 8 & - & - \\ \hline
  \end{tabular}
  \caption{The prediction by $h_1$}
  \label{tab:1st_round_prediction}
  \end{center}
\end{table}

\item[(2)] The distribution $D_i$ over the examples for each round

The distribution $D_i$ evolves over the rounds as shown in Table \ref{tab:distribution_evolution}.

\begin{table}[htb]
  \begin{center}
  \begin{tabular}{|c|c|c|c|c|} \hline
    index & $D_1$ & $D_2$ & $D_3$ \\ \hline
    1 & 0.1 & 0.0625 & 0.125 \\ \hline
    2 & 0.1 & 0.0625 & 0.0417 \\ \hline
    3 & 0.1 & 0.0625 & 0.125 \\ \hline
    4 & 0.1 & 0.0625 & 0.0417 \\ \hline
    5 & 0.1 & 0.0625 & 0.125 \\ \hline
    6 & 0.1 & 0.0625 & 0.125 \\ \hline
    7 & 0.1 & 0.0625 & 0.0417 \\ \hline
    8 & 0.1 & 0.0625 & 0.0417 \\ \hline
    9 & 0.1 & 0.25 & 0.1666 \\ \hline
    10 & 0.1 & 0.25 & 0.1666 \\ \hline
  \end{tabular}
  \caption{The distribution $D_i$}
  \label{tab:distribution_evolution}
  \end{center}
\end{table}

\item[(3)] The final hypothesis after running two rounds.

$H_{\rm final}(x)={\rm sign} \left[ 0.6931 \times {\rm sign}(x_1 > 6) + 0.5493 \times {\rm sign}(x_2 > 9) \right]$

\end{enumerate}

\item {\bf Naive Bayes}

\begin{enumerate}
\item[(1)] Write the joint log-likelihood of a document and labels (i.e., $\log P(D_i, y_i)$).

First, the joint probability $P(d_i, y_i)$ is
\[
\left[ \frac{n!}{a_i!b_i!c_i!d_i!}\alpha_1^{a_i}\beta_1^{b_i}\gamma_1^{c_i}\delta_1^{d_i} \times P(y=1)\right]^y \times \left[ \frac{n!}{a_i!b_i!c_i!d_i!}\alpha_0^{a_i}\beta_0^{b_i}\gamma_0^{c_i}\delta_0^{d_i} \times P(y=0)\right]^{(1-y)}
\]
Since we do not know a prior probability of $P(y)$, we just use the uniform distribution, which is $P(y)=1/2$. Thus,
\[
\frac{1}{2} \times \left[ \frac{n!}{a_i!b_i!c_i!d_i!}\alpha_1^{a_i}\beta_1^{b_i}\gamma_1^{c_i}\delta_1^{d_i}\right]^y \times \left[ \frac{n!}{a_i!b_i!c_i!d_i!}\alpha_0^{a_i}\beta_0^{b_i}\gamma_0^{c_i}\delta_0^{d_i}\right]^{(1-y)}
\]
Then, its logarithm becomes
\begin{eqnarray}
\log \frac{1}{2} + y \log \left[ \frac{n!}{a_i!b_i!c_i!d_i!}\alpha_1^{a_i}\beta_1^{b_i}\gamma_1^{c_i}\delta_1^{d_i}\right] + (1-y) \log \left[ \frac{n!}{a_i!b_i!c_i!d_i!}\alpha_0^{a_i}\beta_0^{b_i}\gamma_0^{c_i}\delta_0^{d_i}\right]
 \nonumber \\
= \log \frac{1}{2} + \log \frac{n!}{a_i!b_i!c_i!d_i!} + y \left[ a_i \log \alpha_1 + b_i \log \beta_1 + c_i \log \gamma_1 + d_i \log \delta_1 \right] \nonumber \\ + (1-y) \left[ a_i \log \alpha_0 + b_i \log \beta_0 + c_i \log \gamma_0 + d_i \log \delta_0 \right] \nonumber
\end{eqnarray}

\end{enumerate}

\item {\bf Bayesian Network}
\begin{enumerate}

\item[(1)] How many parameters are needed to define the network?

\item[(2)] Write the expression calculating (a) $P(A=1, D=2)$ (b) $P(A=1,D=2|C=1)$

(a) $P(A=1)P(D=2)$

(b) $\frac{P(A=1)P(C=1|A=1)P(D=2|C=1)}{P(C=1)}$

\end{enumerate}

\item {\bf Variable Elimination}

\end{enumerate}

\end{document}

